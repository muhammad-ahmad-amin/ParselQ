\documentclass[12pt]{report}

\usepackage{graphicx}
\usepackage{hyperref}
\usepackage{setspace}
\usepackage{titlesec}
\usepackage{geometry}
\usepackage{array}
\usepackage{booktabs}
\geometry{margin=1in}

\hypersetup{
    colorlinks=true,
    linkcolor=blue,
    urlcolor=blue,
    pdfborder={0 0 0}  
}

\titleformat{\chapter}{\normalfont\huge\bfseries}{\thechapter}{1em}{}
\setlength{\parskip}{8pt}

\begin{document}

% ---------------------------------------------------
% TITLE PAGE
% ---------------------------------------------------
\begin{titlepage}
    \centering
    \vspace*{2cm}

    {\Huge \textbf{Assignment 1}}\\[0.5cm]
    {\LARGE \textbf{Dimensional Stance Analysis (ParselQ)}}\\[1cm]

    \vspace{1.5cm}

    {\Large \textbf{Group Members}}\\[0.5cm]
    \large
    Muhammad Ahmad Amin (502217)\\
    Hassan Jamal (519530)\\
    Haniya Farhan (492237)\\
    Syeda Frozish Batool (501165)\\

    \vfill
    \Large Department of Computer Science\\
    Faculty of Computing\\
    National University of Sciences and Technology, Islamabad\\
    \vspace{0.5cm}
    {\large \today}
\end{titlepage}

\tableofcontents
\newpage


% ---------------------------------------------------
% INTRODUCTION
% ---------------------------------------------------
\chapter{Introduction}

This report presents the comprehensive Exploratory Data Analysis (EDA) for the project \textbf{Dimensional Stance Analysis (ParselQ)}, developed as part of Assignment 1.  
The task focuses on understanding stance variations in an environmental protection dataset under Track B, Subtask 1.

To complete the EDA efficiently, the work was divided equally among four group members. Each member was responsible for one major component as shown in the table below:

\section*{Task Division Table}
\begin{center}
\begin{tabular}{|p{4cm}|p{6cm}|p{3cm}|}
\hline
\textbf{Member} & \textbf{Assigned Task} & \textbf{CMS ID} \\ \hline
Muhammad Ahmad Amin & Textual Token-Level + Valence - Arousal Label Analysis& 502217 \\ \hline
Hassan Jamal & Dataset Overview + Structure Validation& 519530 \\ \hline
Haniya Farhan & Aspect Frequency + Label Behavior Analysis & 492237 \\ \hline
Syeda Frozish Batool & Latex Report& 501165 \\ \hline
\end{tabular}
\end{center}

The remaining chapters provide detailed explanations and allocated plot spaces for each part.

\newpage



% ---------------------------------------------------
% PART 1
% ---------------------------------------------------
\chapter{Part 1: Dataset Structure, Validation, and Overview}

\section{Detailed Explanation}

This component establishes the foundation of the entire EDA by examining the raw dataset structure.  
The dataset is provided in JSONL format, where each line represents a post containing textual content and a list of annotated stance aspects.  

The key goals of this part include:  
\begin{itemize}
    \item Examining structural integrity of the dataset  
    \item Identifying missing or malformed entries  
    \item Computing essential statistics such as total aspects, average aspects per post, and missing fields  
    \item Performing consistency checks for format, field types, and annotation validity  
    \item Manually inspecting samples to ensure annotations align with text meaning  
\end{itemize}

This structural analysis ensures that later EDA steps do not suffer from hidden dataset inconsistencies, which can corrupt model training.
\pagebreak


\section{Plots}

% ------- PLOT BOX 1 -------
\vspace{0.5cm}
\noindent
\begin{center}
\centering
\includegraphics[width=0.8\textwidth]{plots/dataset_size.pdf}
\end{center}

% ------- PLOT BOX 2 -------
\vspace{0.5cm}
\noindent
\begin{center}
\centering
\includegraphics[width=0.8\textwidth]{plots/zero_vs_nonzero_train.pdf}
\end{center}

\newpage


% ---------------------------------------------------
% PART 2
% ---------------------------------------------------
\chapter{Part 2: Textual Token Analysis}

\section{Detailed Explanation}

Linguistic analysis is essential for understanding the nature of textual content in stance detection tasks.  
This part examines the token-level structure of the posts using NLTK.  

The analyses conducted include:  
\begin{itemize}
    \item **Token length distribution:** identifies short vs. long posts and outliers  
    \item **Vocabulary richness:** measures lexical uniqueness and domain specificity  
    \item **Stopword frequency:** reveals conversational vs. formal tendencies  
    \item **N-gram extraction:** uncovers commonly co-occurring terms hinting at stance framing  
\end{itemize}

These insights guide preprocessing decisions such as stemming, removal of stopwords, and handling of rare words.

\pagebreak


\section{Plots}

% ------- PLOT BOX 1 -------
\vspace{0.5cm}
\noindent
\begin{center}
\centering
\includegraphics[width=0.9\textwidth]{plots/token_length_distribution.pdf}
\end{center}

% ------- PLOT BOX 2 -------
\vspace{0.5cm}
\noindent
\begin{center}
\centering
\includegraphics[width=0.9\textwidth]{plots/top_tokens.pdf}
\end{center}

\newpage


% ---------------------------------------------------
% PART 3
% ---------------------------------------------------
\chapter{Part 3: Aspect Distribution and Label Behavior}

\section{Detailed Explanation}

This segment dives into the stance aspects themselves — their frequency, distribution, and relationships.  
This analysis is crucial because label imbalance directly affects classification performance.

The following analyses were performed:  
\begin{itemize}
    \item **Aspect frequency distribution:** to identify dominant stance aspects  
    \item **Skewness/imbalance detection:** to determine need for resampling or class weights  
    \item **Aspect co-occurrence:** reveals relationships and frequently paired stance dimensions  
    \item **Summary statistics:** aspects/post, variance, and distribution shape  
\end{itemize}

Understanding label behavior ensures a fair and robust downstream ML model.
\pagebreak

\section{Plots}

% ------- PLOT BOX 1 -------
\vspace{0.5cm}
\noindent
\begin{center}
\centering
\includegraphics[width=0.8\textwidth]{plots/va_vs_length.pdf}
\end{center}

% ------- PLOT BOX 2 -------
\vspace{0.5cm}
\noindent
\begin{center}
\centering
\includegraphics[width=0.7\textwidth]{plots/aspect_histogram.pdf}
\end{center}

\newpage


% ---------------------------------------------------
% PART 4
% ---------------------------------------------------
\chapter{Part 4: Valence - Arousal Label Analysis}

\section{Detailed Explanation}

This part focuses on the valence and arousal (VA) labels, which are available **only in the training set**.  
Visualizing these labels helps understand the emotional distribution of the dataset.  

The insights include:  
\begin{itemize}
    \item Distribution of Valence scores  
    \item Distribution of Arousal scores  
    \item Relationship between Valence and Arousal (scatter plot)  
    \item Correlation between Valence and Arousal  
\end{itemize}

These observations highlight whether the VA labels are balanced or skewed, whether texts are generally positive or negative, and the prevalence of high-arousal posts.  
Such insights are crucial for designing models that effectively capture affective dimensions.
\pagebreak

\section{Plots}

\vspace{0.5cm}
\noindent
\begin{center}
\centering
\includegraphics[width=0.6\textwidth]{plots/va_scatter.pdf}
\end{center}

\vspace{0.5cm}
\noindent
\begin{center}
\centering
\includegraphics[width=0.7\textwidth]{plots/arousal_distribution.pdf}
\end{center}


% ---------------------------------------------------
% GITHUB LINK
% ---------------------------------------------------
\chapter*{GitHub Repository}
\addcontentsline{toc}{chapter}{GitHub Repository}


\vspace{1cm}
\begin{center}
\textbf{\Large \underline{(https://github.com/muhammad-ahmad-amin/ParselQ.git)}}
\end{center}

\vspace{1cm}

\begin{center}
\Large Thank You!
\end{center}

\end{document}