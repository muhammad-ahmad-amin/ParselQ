\documentclass[12pt]{report}

\usepackage{graphicx}
\usepackage{hyperref}
\usepackage{setspace}
\usepackage{titlesec}
\usepackage{geometry}
\usepackage{array}
\usepackage{booktabs}
\usepackage{float}
\usepackage{caption}
\geometry{margin=1in}

\hypersetup{
    colorlinks=true,
    linkcolor=blue,
    urlcolor=blue,
    pdfborder={0 0 0}  
}

\titleformat{\chapter}{\normalfont\huge\bfseries}{\thechapter}{1em}{}
\setlength{\parskip}{8pt}

\begin{document}

% ---------------------------------------------------
% TITLE PAGE
% ---------------------------------------------------
\begin{titlepage}
    \centering
    \vspace*{2cm}

    {\Huge \textbf{Assignment 2}}\\[0.5cm]
    {\LARGE \textbf{Baseline Modeling for Dimensional Stance Analysis (ParselQ)}}\\[1cm]

    \vspace{1.5cm}

    {\Large \textbf{Group Members}}\\[0.5cm]
    \large
    Muhammad Ahmad Amin (502217)\\
    Hassan Jamal (519530)\\
    Haniya Farhan (492237)\\
    Syeda Frozish Batool (501165)\\

    \vfill
    \Large Department of Computer Science\\
    Faculty of Computing\\
    National University of Sciences and Technology, Islamabad\\
    \vspace{0.5cm}
    {\large \today}
\end{titlepage}

\tableofcontents
\newpage



% ---------------------------------------------------
% CHAPTER 1: INTRODUCTION
% ---------------------------------------------------
\chapter{Introduction}

This report presents the complete baseline modeling and experimental framework for the project \textbf{Dimensional Stance Analysis (ParselQ)} under Track B, Subtask 1.  
The objective is to build and evaluate different machine learning and deep learning baselines for stance prediction along multiple dimensions.

In Assignment 1, the focus was Exploratory Data Analysis (EDA).  
In Assignment 2, the goal shifts toward:

\begin{itemize}
    \item Designing baseline architectures.
    \item Creating preprocessing pipelines.
    \item Implementing and training the models.
    \item Comparing performance across classical ML and Transformer-based baselines.
    \item Reporting implementation details and runtime observations.
\end{itemize}

The following baselines were developed:

\begin{enumerate}
    \item \textbf{Baseline A: TF-IDF + Logistic Regression}
    \item \textbf{Baseline B: Bi-LSTM (Neural RNN Model)}
    \item \textbf{Baseline C: BERT-base (Transformer Fine-Tuning)}
    \item \textbf{Baseline D: DistilBERT (Lightweight Transformer Model)}
\end{enumerate}

Each model includes:

\begin{itemize}
    \item Preprocessing pipeline
    \item Model architecture description
    \item Implementation details
\end{itemize}



% ---------------------------------------------------
% TASK DIVISION
% ---------------------------------------------------
\chapter{Task Division}

\begin{center}
\begin{tabular}{|p{4cm}|p{7cm}|p{3cm}|}
\hline
\textbf{Member} & \textbf{Assigned Baseline Task (Assignment 2)} & \textbf{CMS ID} \\ \hline
Muhammad Ahmad Amin & Baseline Model A (TF-IDF + Logistic Regression) + data_loader.py + preprocess.py& 502217 \\ \hline
Hassan Jamal & Baseline Model B (Bi-LSTM) + train.py + LaTeX Compilation& 519530 \\ \hline
Haniya Farhan & Baseline Model C (BERT-base Fine-tuning) & 492237 \\ \hline
Syeda Frozish Batool & Baseline Model D (DistilBERT / RoBERTa-small) & 501165 \\ \hline
\end{tabular}
\end{center}

\newpage




% ---------------------------------------------------
% CHAPTER 3: BASELINE MODEL A
% ---------------------------------------------------
\chapter{Baseline Model A: TF-IDF + Logistic Regression}

\section{Overview}

Baseline A represents the most fundamental machine learning pipeline and is used to establish a lower bound for performance.  
The goal of this baseline is to understand how far a purely statistical, surface-level model can go without learning contextual semantics.  
It relies entirely on TF-IDF features, meaning the model only captures word importance rather than deeper linguistic relationships.

This baseline is useful because it is lightweight, interpretable, requires minimal training time, and provides quick feedback about data quality.  
It also helps compare how much improvement advanced models provide over simple linear decision boundaries.

\section{Pipeline Diagram (Placeholder)}
\begin{figure}[H]
    \centering
    \fbox{\includegraphics[width=0.8\textwidth]{pieline_model_A.png}}
    \caption{Pipeline for Baseline A (Insert diagram here)}
\end{figure}

\section{Model Architecture}

\begin{itemize}
    \item TF-IDF vectorizer (unigram + bigram features)
    \item Logistic Regression classifier (one-vs-rest)
\end{itemize}
\section{Results and Visualization}

\subsection{Performance Plot (Placeholder)}
\begin{figure}[H]
    \centering
    \vspace{0.5cm}
    \fbox{
        \includegraphics[width=0.75\textwidth]{baselineA_results.pdf}
    }
    \caption{Performance plot for Baseline A }
\end{figure}

\newpage




% ---------------------------------------------------
% CHAPTER 4: BASELINE MODEL B — BI-LSTM
% ---------------------------------------------------
\chapter{Baseline Model B: Bi-LSTM (Recurrent Neural Network)}

\section{Overview}

Baseline B leverages a bidirectional LSTM, allowing the model to read text both forward and backward.  
This enables the network to capture sequential patterns, long-range dependencies, and contextual clues that traditional bag-of-words models cannot detect.

Unlike TF-IDF, which treats each sentence as a unordered collection of words, the Bi-LSTM learns semantic patterns, sentence structure, and token interactions over time.  
This makes it especially useful for stance detection, where the ordering of words such as “not”, “however”, or “but” significantly impacts meaning.

The model aims to provide a stronger deep learning baseline before transitioning to transformer-based architectures.

\section{Pipeline Diagram (Placeholder)}

\begin{figure}[H]
    \centering
    \fbox{\includegraphics[width=0.8\textwidth]{pieline_model_B.png}}
    \caption{Pipeline for Baseline B (Insert diagram here)}
\end{figure}

\section{Model Architecture}

\begin{itemize}
    \item Tokenization + padding
    \item Embedding layer
    \item Bidirectional LSTM
    \item Dropout regularization
    \item Dense layer + softmax output
\end{itemize}

\section{Implementation Notes}

\begin{itemize}
    \item Long sequences may require truncation
    \item Learning rates must be tuned carefully to avoid instability
    \item GPU training significantly accelerates model convergence
\end{itemize}
\section{Results and Visualization}

\subsection{Performance Plot (Placeholder)}
\begin{figure}[H]
    \centering
    \vspace{0.5cm}
    \fbox{
        \includegraphics[width=0.75\textwidth]{baseline_B_report.pdf}
    }
    \caption{Performance plot for Baseline B }
\end{figure}

\newpage




% ---------------------------------------------------
% CHAPTER 5: BASELINE MODEL C — BERT-BASE
% ---------------------------------------------------
\chapter{Baseline Model C: BERT-base Fine-Tuning}

\section{Overview}

Baseline C uses BERT-base, a 12-layer bidirectional Transformer model that learns deep contextual representations.  
Unlike RNNs or TF-IDF-based models, BERT processes entire sentences simultaneously using self-attention, allowing it to understand relationships between tokens regardless of distance.

This baseline is expected to produce major performance improvements, especially for tasks requiring subtle stance differentiation.  
BERT captures sentiment shifts, sarcasm cues, topic relevance, and implicit opinions, making it ideal for nuanced classification problems like stance analysis.

It also benefits from pretraining on massive corpora, enabling strong generalization even with limited labeled data.

\section{Pipeline Diagram (Placeholder)}

\begin{figure}[H]
    \centering
    \fbox{\includegraphics[width=0.8\textwidth]{pieline_model_C.png}}
    \caption{Pipeline for Baseline C (Insert diagram here)}
\end{figure}

\section{Model Architecture}

\begin{itemize}
    \item BERT tokenizer
    \item Input IDs + attention masks
    \item Pretrained BERT-base encoder
    \item CLS pooled output → fully connected layer
\end{itemize}

\section{Implementation Notes}

\begin{itemize}
    \item GPU recommended due to heavy computation
    \item Layer freezing can speed training
    \item Proper mask handling crucial for padded inputs
\end{itemize}

\newpage




% ---------------------------------------------------
% CHAPTER 6: BASELINE MODEL D — DISTILBERT
% ---------------------------------------------------
\chapter{Baseline Model D: DistilBERT (Light Transformer Baseline)}

\section{Overview}

Baseline D implements DistilBERT, a distilled and compressed version of BERT that retains almost the same accuracy while being significantly faster.  
It is designed to provide a middle ground between accuracy and computational efficiency, making it suitable for deployments or rapid experimentation.

DistilBERT removes the larger BERT’s redundant layers while keeping most of its representational power.  
This baseline tests whether a lightweight transformer can still achieve strong stance classification performance without the full computational overhead.

It is especially useful when training time, hardware constraints, or memory limitations are major considerations.

\section{Pipeline Diagram (Placeholder)}

\begin{figure}[H]
    \centering
    \fbox{\includegraphics[width=0.8\textwidth]{pieline_model_D.png}}
    \caption{Pipeline for Baseline D (Insert diagram here)}
\end{figure}

\section{Model Architecture}

\begin{itemize}
    \item DistilBERT tokenizer
    \item 6-layer transformer encoder
    \item Classification head
\end{itemize}








\section{Conclusion}

This assignment demonstrated the progressive improvement from classical ML models to advanced transformer-based baselines.  
Future work includes hyperparameter tuning, data augmentation, and multi-label stance analysis.

% ---------------------------------------------------
% GITHUB REPOSITORY
% ---------------------------------------------------
\chapter*{GitHub Repository}
\addcontentsline{toc}{chapter}{GitHub Repository}

\begin{center}
\fbox{\href{https://github.com/your-repo-link}{https://github.com/muhammad-ahmad-amin/ParselQ.git}}
\end{center}

\end{document}
